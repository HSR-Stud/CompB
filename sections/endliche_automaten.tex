\section{Endliche Automaten}


\subsection{Umwandlung NFA $\rightarrow$ CFG}

Ein regulärer Ausdruck in Form eines endlichen Automaten (NFA) kann auch als Kontextfreie Grammatik
(CFG) dargestellt werden:

\begin{itemize}
	\item Jeder Zustand des NFA wird zu einem Nicht-Terminal $A_i$
	\item Falls $i$ der Startzustand ist, wird $A_i$ das Startsymbol
	\item Ein Zustandsübergang $i \rightarrow j$ bei Eingabe von $a$ wird zur Produktion $A_i
		\rightarrow aA_j$ (bei $\varepsilon$-Produktionen $A_i \rightarrow A_j$)
	\item Falls $i$ ein Akzeptierzustand ist: $A_i \rightarrow \varepsilon$
\end{itemize}

Nachfolgend eine Beispielumwandlung:

\subsubsection*{Regex}

$(a|b)*abb$

\subsubsection*{NFA}

\begin{tikzpicture}[->,>=stealth',shorten >=1pt,auto,node distance=2.8cm,semithick]

	\node[state,initial]   (Q1)               {$q_1$};
	\node[state]           (Q2) [right of=Q1] {$q_2$};
	\node[state]           (Q3) [right of=Q2] {$q_3$};
	\node[state,accepting] (Q4) [right of=Q3] {$q_4$};

	\path (Q1) edge [loop below] node {$a,b$} (Q1)
	      (Q1) edge              node {$a$}   (Q2)
	      (Q2) edge              node {$b$}   (Q3)
	      (Q3) edge              node {$b$}   (Q4);

\end{tikzpicture}

\subsubsection*{CFG}

\begin{flalign*}
	A_0 & \rightarrow aA_0 \mid bA_0 \mid aA_1 &\\
	A_1 & \rightarrow bA_2 &\\
	A_2 & \rightarrow bA_3 &\\
	A_3 & \rightarrow \varepsilon &
\end{flalign*}
