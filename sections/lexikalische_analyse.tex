\section{Lexikalische Analyse}


\subsection{Aufgaben eines Lexers}

Der Lexer liesst den Eingabestromes und unterteilt ihn in Lexeme. Zusätzlich
trägt er Bezeichner in eine Symboltabelle ein.

Wichtig sind dabei folgende Begriffe:

\begin{description}
	\item[Lexem] Lexeme sind Zeichenketten, welche die kleinsten syntaktischen
		Einheiten einer Programmiersprache bilden.
	\item[Token] Ein Token besteht aus einem Namen und einem optionalen
		Attributwert. Tokens formen eine syntaktische Kategorie für eine Klasse von
		Lexemen. Mögliche Tokens sind Schlüsselwörter, Bezeichner, Operatoren oder
		Konstanten.
	\item[Pattern] Ein Pattern beschreibt den Aufbau eines Lexems eines Tokens.
		Häufig werden die Patterns als Reguläre Ausdrücke dargestellt.
	\item[Symboltabelle] In der Regel werden die Tokens nicht als Text
		referenziert, sondern als Symbol/Zahl. Die Symboltabelle enthält Mappings
		von Symbolen zu Tokens.
\end{description}

\textbf{Beispiel}

Aus dem Programmcode rechts kann man die Lexeme und Tokens links
herauslesen.

\begin{minipage}{.4\linewidth}
	\begin{tabular}[h]{|l|l|}
		\hline
		\textbf{Lexem} & \textbf{Token} \\
		\hline
		x & ID \\
		= & ASSIGNMENT \\
		foo & ID \\
		- & ARITHMETIC \\
		1 & NUMBER \\
		; & SEMICOLON \\
		\hline
	\end{tabular}
\end{minipage}
\begin{minipage}{.6\linewidth}
	\large\texttt{x = foo - 1;}
\end{minipage}


\subsection{Probleme, Schwerigkeiten}

Viele Sprachkonstrukte sind in der Phase der Lexikalischen Analyse noch nicht
eindeutig identifizierbar. Die Zeichenkette \texttt{fi} kann zum Beispiel als
Identifier oder als falsch geschriebenes Schlüsselwort \texttt{if} auftreten.
Dadurch ist es für einen Lexer auch sehr Schwer, Fehler im Programm zu erkennen.

Zudem spielt auch die Verarbeitungsreihenfolge eine wichtige Rolle. Werden
beispielsweise Identifier vor den Schlüsselwörtern verarbeitet, so wird
\texttt{while} als Variablenname und nicht als Schlüsselwort registriert.
