\section{Lokale Optimierungen}

\subsection{Algebraic Simplifications}

Einige algebraische Anweisungen können vereinfacht werden. Beispiel:

\begin{varwidth}{\textwidth}\ttfamily
X := 1 * A\\
Y := X ** 2\\
Q := Y + 0\\
R := Q * 8
\end{varwidth}%
\hspace{1cm}$\Rightarrow$\hspace{1cm}%
\begin{varwidth}{\textwidth}\ttfamily
X := A\\
Y := X * X\\
Q := Y\\
R := Q << 3
\end{varwidth}

\subsection{Constant Folding}

Ausdrücke mit konstanten Werten werden zur Compile-Zeit evaluiert. Beispiel:

\begin{varwidth}{\textwidth}\ttfamily
X := 3 * 2 + 4
\end{varwidth}%
\hspace{1cm}$\Rightarrow$\hspace{1cm}%
\begin{varwidth}{\textwidth}\ttfamily
X := 10
\end{varwidth}

\subsection{Constant Propagation}

Konstanten werden wo immer möglich eingesetzt. Beispiel:

\begin{varwidth}{\textwidth}\ttfamily
X := 3\\
Y := Z * W\\
Q := X + Y
\end{varwidth}%
\hspace{1cm}$\Rightarrow$\hspace{1cm}%
\begin{varwidth}{\textwidth}\ttfamily
X := 3\\
Y := Z * W\\
Q := 3 + Y
\end{varwidth}

\subsection{Copy Propagation}

Falls in einem Block \texttt{W := X} vorkommt, kann man in allen folgenden
Einsätzen von \texttt{W} anstelle von \texttt{W} gleich \texttt{X} verwenden.
Beispiel:

\begin{varwidth}{\textwidth}\ttfamily
X := 7\\
Y := X\\
Q := A + X
\end{varwidth}%
\hspace{1cm}$\Rightarrow$\hspace{1cm}%
\begin{varwidth}{\textwidth}\ttfamily
X := 7\\
Y := 7\\
Q := A + 7
\end{varwidth}

\subsection{Dead Code Elimination}

Ungenutzer Code wird entfernt. Beispiel:

\begin{varwidth}{\textwidth}\ttfamily
X := 3\\
Y := Z * W\\
Q := 3 + Y
\end{varwidth}%
\hspace{1cm}$\Rightarrow$\hspace{1cm}%
\begin{varwidth}{\textwidth}\ttfamily
Y := Z * W\\
Q := 3 + Y
\end{varwidth}
