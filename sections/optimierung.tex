\section{Optimierung}

\subsection{Lokale / Globale Optimierung}

\textbf{Lokale Optimierung} ist unabhängig vom globalen Zustand eines
Programmes.

\textbf{Globale Optimierung} hängt von der Kenntnis einer Eigenschaft $P$ zu
einem bestimmten Zeitpunkt der Programmausführung ab. Damit die Eigenschaft $P$
zu \textit{jedem} Ausführungszeitpunkt überprüft werden kann, muss das gesamte
Programm bekannt sein. Aus Sicherheitsgründen sollte man deshalb eher
konservativ optimieren. Im Zweifelsfall ist es besser nichts zu tun.

\subsection{Algebraic Simplifications}

Einige algebraische Anweisungen können vereinfacht werden. Beispiel:

\begin{varwidth}{\textwidth}\ttfamily
X := 0 * A\\
Y := B ** 2\\
Q := C * 8
\end{varwidth}%
\hspace{1cm}$\Rightarrow$\hspace{1cm}%
\begin{varwidth}{\textwidth}\ttfamily
X := 0\\
Y := B * B\\
Q := C \verb|<<| 3
\end{varwidth}

\subsection{Constant Folding}

Ausdrücke mit konstanten Werten werden zur Compile-Zeit evaluiert. Beispiel:

\begin{varwidth}{\textwidth}\ttfamily
X := 3 * 2 + 4
\end{varwidth}%
\hspace{1cm}$\Rightarrow$\hspace{1cm}%
\begin{varwidth}{\textwidth}\ttfamily
X := 10
\end{varwidth}

\subsection{Constant Propagation}

Konstanten werden wo immer möglich eingesetzt. Beispiel:

\begin{varwidth}{\textwidth}\ttfamily
X := 3\\
Q := X + Y
\end{varwidth}%
\hspace{1cm}$\Rightarrow$\hspace{1cm}%
\begin{varwidth}{\textwidth}\ttfamily
X := 3\\
Q := 3 + Y
\end{varwidth}

\subsection{Copy Propagation}

Falls in einem Block \texttt{W := X} vorkommt, kann man in allen folgenden
Einsätzen von \texttt{W} anstelle von \texttt{W} gleich \texttt{X} verwenden.
Beispiel:

\begin{varwidth}{\textwidth}\ttfamily
X := 7\\
Q := A + X
\end{varwidth}%
\hspace{1cm}$\Rightarrow$\hspace{1cm}%
\begin{varwidth}{\textwidth}\ttfamily
X := 7\\
Q := A + 7
\end{varwidth}

\subsection{Dead Code Elimination}

Ungenutzer Code wird entfernt. Beispiel:

\begin{varwidth}{\textwidth}\ttfamily
X := 3\\
Y := Z * W\\
Q := 3 + Y
\end{varwidth}%
\hspace{1cm}$\Rightarrow$\hspace{1cm}%
\begin{varwidth}{\textwidth}\ttfamily
Y := Z * W\\
Q := 3 + Y
\end{varwidth}
