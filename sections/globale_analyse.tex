\section{Globale Analyse}


\subsection{Datenflussanalyse}

Die Datenflussanalyse verknüpft mit jedem Programmpunkt einen Datenflusswert.
Die Datenflusswerte vor und hinter einer Anwendung $p$ sind $IN[p]$ und
$OUT[p]$. Die Relation zwischen $IN$ und $OUT$ wird als
\textit{Transferfunktion} bezeichnet.

Innerhalb eines Grundblocks mit den Anweisungen $s_1,\ldots,s_n$ gilt:
$IN[s_{i+1}] = OUT[s_i]$.


\subsection{Liveness-Analyse}

\textbf{Regeln}

\begin{enumerate}
	\item Eine Variable \textbf{nach} einem Statement ist lebendig (alive), wenn
		sie \textbf{vor} einem nachfolgenden Statement lebendig ist.\\
		\[ L(p, x, out) = \lor \{ L(s, x, in) | s \textrm{ is a successor of } p \} \]
	\item Wenn eine Variable in einem Statement gelesen wird, ist sie \textbf{vor}
		dem Statement lebendig.\\
		\[ L(s, x, in) = true \textrm{ if } s \textrm{ refers to } x \textrm{ on the rhs} \]
	\item Wenn eine Variable in einem Statement geschrieben, aber nicht gelesen
		wird, ist sie \textbf{vor} dem Statement tot (dead).\\
		\[ L(x := e, x, in) = false \textrm{ if } e \textrm{ does not refer to } x \]
	\item Wenn in einem Statement eine Variable weder gelesen noch geschrieben
		wird, ist die Liveness \textbf{vor} dem Statement gleich wie \textbf{nach}
		dem Statement.\\
		\[ L(s, x, in) = L(s, x, out) \textrm{ if } s \textrm{ does not refer to } x \]
\end{enumerate}

\textbf{Algorithmus}

\begin{enumerate}
	\item Zu Beginn ist die Liveness überall $false$.
	\item Wähle ein Statement $s$, welches eine der Regeln 1-4 nicht erfüllt.
		Korrigiere die Liveness. Wiederhole diesen Schritt bis alle Statements die
		Regeln 1-4 erfüllen.
\end{enumerate}

Jede Stelle kann bei diesem Algorithmus höchstens ein mal die Liveness wechseln,
zudem kann die Liveness nur von $false$ zu $true$ wechseln, nicht umgekehrt.
