\section{Globale Optimierungen}

\subsection{Risiken}

Globale Optimierung hängt von der Kenntnis einer Eigenschaft $P$ zu einem
bestimmten Zeitpunkt der Programmausführung ab. Damit die Eigenschaft $P$ zu
\textit{jedem} Ausführungszeitpunkt überprüft werden kann, muss das gesamte
Programm bekannt sein.

Aus Sicherheitsgründen sollte man deshalb eher konservativ optimieren. Im
Zweifelsfall ist es besser nichts zu tun.

\subsection{Datenflussanalyse}

Die Datenflussanalyse verknüpft mit jedem Programmpunkt einen Datenflusswert.
Die Datenflusswerte vor und hinter einer Anwendung $p$ sind $IN[p]$ und
$OUT[p]$.

Die Relation zwischen $IN$ und $OUT$ wird als \textit{Transferfunktion}
bezeichnet.

Innerhalb eines Grundblocks mit den Anweisungen $s_1,\ldots,s_n$ gilt:
$IN[s_{i+1}] = OUT[s_i]$.
