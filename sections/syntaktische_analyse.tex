\section{Syntaktische Analyse}


\subsection{Umwandlung NFA $\rightarrow$ CFG}

Ein regulärer Ausdruck in Form eines endlichen Automaten (NFA) kann auch als Kontextfreie Grammatik
(CFG) dargestellt werden:

\begin{itemize}
	\item Jeder Zustand des NFA wird zu einem Nicht-Terminal $A_i$
	\item Falls $i$ der Startzustand ist, wird $A_i$ das Startsymbol
	\item Ein Zustandsübergang $i \rightarrow j$ bei Eingabe von $a$ wird zur Produktion $A_i
		\rightarrow aA_j$ (bei $\varepsilon$-Produktionen $A_i \rightarrow A_j$)
	\item Falls $i$ ein Akzeptierzustand ist: $A_i \rightarrow \varepsilon$
\end{itemize}

Nachfolgend eine Beispielumwandlung:

\subsubsection*{Regex}

$(a|b)*abb$

\subsubsection*{NFA}

\begin{tikzpicture}[->,>=stealth',shorten >=1pt,auto,node distance=2.8cm,semithick]

	\node[state,initial]   (Q1)               {$q_1$};
	\node[state]           (Q2) [right of=Q1] {$q_2$};
	\node[state]           (Q3) [right of=Q2] {$q_3$};
	\node[state,accepting] (Q4) [right of=Q3] {$q_4$};

	\path (Q1) edge [loop below] node {$a,b$} (Q1)
	      (Q1) edge              node {$a$}   (Q2)
	      (Q2) edge              node {$b$}   (Q3)
	      (Q3) edge              node {$b$}   (Q4);

\end{tikzpicture}

\subsubsection*{CFG}

\begin{flalign*}
	A_0 & \rightarrow aA_0 \mid bA_0 \mid aA_1 &\\
	A_1 & \rightarrow bA_2 &\\
	A_2 & \rightarrow bA_3 &\\
	A_3 & \rightarrow \varepsilon &
\end{flalign*}


\subsection{Eliminieren von Linksrekursion}

Einfache (direkte) Linksrekursionen können mithilfe des folgenden Algorithmus einfach entfernt
werden:

\begin{itemize}
	\item Es seien $A \rightarrow A\alpha_1 \mid A\alpha_2 \mid \ldots \mid A\alpha_m$ die
		linksrekursiven Regeln
	\item Es seien $A \rightarrow \beta_1 \mid \beta_2 \mid \ldots \mid \beta_n$ die nicht
		linksrekursiven Regeln
	\item Ersetze die erste Gruppe durch $A' \rightarrow \alpha_1A'\mid \alpha_2A' \mid \ldots \mid
		\alpha_mA'$
	\item Ersetze die zweite Gruppe durch $A \rightarrow \beta_1A' \mid \beta_2A' \mid \ldots \mid
		\beta_nA'$
	\item Füge folgende Regel hinzu: $A' \rightarrow \varepsilon$
\end{itemize}

Hinweis: Die Grammatik darf keine $\varepsilon$-Produktionen enthalten, ansonsten kann eine
versteckte Linksrekursion auftreten.

\paragraph{Beispiel}

\begin{flalign*}
	A & \rightarrow Aa \mid bA \mid c &
\end{flalign*}
\ldots{wird} zu
\begin{flalign*}
	A  & \rightarrow bAA' \mid cA' &\\
	A' & \rightarrow aA' \mid \varepsilon &
\end{flalign*}


\subsection{FIRST und FOLLOW Sets}

\subsubsection{FIRST Set}

Das FIRST-Set für ein Nicht-Terminal X besteht aus allen Terminalsymbolen, die auf X folgen können.
Dabei werden nachfolgende Terminalsymbole aufgelöst.

Man betrachtet dafür alle Produktionen, wo X auf der linken Seite steht:

% TODO Punkte 3 und 4 sind noch schlecht formuliert. Das muss noch besser gehen!
\begin{enumerate}
	\item Wenn auf der rechten Seite der Produktion als erstes ein Terminal steht, gehört dieses
		zum FIRST-Set von X.
	\item Wenn auf der rechten Seite der Produktion ein $\varepsilon$ steht, gehört dieses zum
		FIRST-Set von X.
	\item Wenn auf der rechten Seite der Produktion als erstes ein Nicht-Terminal Y steht, gehört das
		FIRST-Set von Y (abzüglich allfälliger $\varepsilon$) zum FIRST-Set von X.
	\item Wenn dieses FIRST-Set von Y ein $\varepsilon$ enthält, berücksichtigt man auch alle Fälle,
		bei welchen $\varepsilon$ weggelassen wird. Wenn es keine solchen Fälle gibt, gehört auch
		$\varepsilon$ zum FIRST-Set von X.
\end{enumerate}

\subsubsection*{Beispiel}

Grammatik:
\begin{flalign*}
	S & \rightarrow Ax \mid By \mid z &\\
	A & \rightarrow 1CB \mid 2CB &\\
	B & \rightarrow 3B \mid C &\\
	C & \rightarrow 4 \mid \varepsilon &
\end{flalign*}
Lösungsweg:
\begin{enumerate}
	\item Das FIRST-Set von $C$ ist $\{4,\varepsilon\}$ gemäss Regeln 1 und 2.
	\item Das FIRST-Set von $B$ enthält gemäss Regel 1 $\{3\}$.
	\item Aufgrund der Produktion $B \rightarrow C$ und der Regel 3 enthält das FIRST-Set von $B$ auch
		das FIRST-Set von $C$, abzüglich des $\varepsilon$.
	\item Aufgrund der Regel 4 werden auch alle Produktionen berücksichtigt, bei denen das
		$\varepsilon$ (und somit das $C$) ``überlesen'' wird. Da auf $C$ jedoch nichts mehr folgt,
		enthält FIRST(B) auch $\{\varepsilon\}$.
	\item Das FIRST-Set von $A$ ist gemäss Regel 1 $\{1,2\}$.
	\item Das FIRST-Set von $S$ enthält gemäss Regel 1 $\{z\}$.
	\item Das FIRST-Set von $S$ enthält gemäss Regel 3 das FIRST-Set von $A$ und $B$, abzüglich
		des $\varepsilon$ in FIRST(B).
	\item Da FIRST(B) $\varepsilon$ enthält, wird zusätzlich der Fall berücksichtigt, bei dem $B$
		weggelassen wird. Da in der Produktion $S \rightarrow By$ auf das $B$ ein $y$ folgt, ist auch
		dieses in FIRST(S) enthalten. (Falls anstelle von $y$ ein Non-Terminal folgen würde, müsste man
		auch dieses auflösen.)
\end{enumerate}
Obige Lösungsschritte resultieren in folgenden FIRST-Sets:
\begin{flalign*}
	& FIRST(S) = \{1,2,3,4,y,z\} &\\
	& FIRST(A) = \{1,2\} &\\
	& FIRST(B) = \{3,4,\varepsilon\} &\\
	& FIRST(C) = \{4,\varepsilon\} &
\end{flalign*}
