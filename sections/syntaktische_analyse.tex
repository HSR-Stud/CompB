\section{Syntaktische Analyse}


\subsection{Aufgaben eines Parsers}

Die Aufgabe des Parser besteht darin, die Eingabesymbole, die der Lexer liefert,
auf Korrektheit bezüglich der Grammatik der Sprache zu überprüfen.

Für korrekte Eingaben erstellt der Parser einen Syntaxbaum, welcher
anschliessend weiterverarbeitet wird. Bei Syntaxfehlern sollte der Parser
möglichst gute Fehlermeldungen liefern und falls möglich mit der Verarbeitung
der Eingabe fortfahren.

Es gibt zwei Typen von Parsern: Top-Down und Bottom-Up Parser. Der Unterschied
besteht darin, wie die Parse-Bäume aufgebaut werden, entweder von den Blättern
zur Wurzel (Bottom-Up) oder umgekehrt (Top-Down).


\subsection{Grammatik}

Eine Grammatik besteht aus folgenden vier Elementen:

\begin{itemize}
	\item Einer Menge Terminalsymbole (\textbf{Tokens}).
	\item Einer Menge Nicht-Terminalsymbole (\textbf{syntaktische Variablen}), aus
		welchen sich Terminalsymbole ableiten lassen.
	\item Einer Menge von Regeln (\textbf{Produktionen}). Auf der linken Seite
		einer Produktion steht ein Nicht-Terminalsymbol, auf der Rechten eine Folge
		von Terminalsymbolen und Nicht-Terminalsymbolen.
	\item Einem \textbf{Startsymbol}.
\end{itemize}

Nachfolgend eine Beispielgrammatik. Terminalsymbole sind in Kleinbuchstaben
geschrieben, Nicht-Terminalsymbole mit Grossbuchstaben. Die erste Regel hat das
Startsymbol auf der Linken Seite.
%
\begin{align*}
	1.~E &\rightarrow E + T \\
	2.~E &\rightarrow T \\
	3.~T &\rightarrow (E) \\
	4.~T &\rightarrow id
\end{align*}

Es gibt verschiedene Typen von Grammatiken. Diese werden normalerweise mit
$X_1X_2(k)$ bezeichnet:
\begin{itemize}
	\item $X_1$: Leserichtung. Entweder $L$ (links-nach-rechts) oder $R$ (rechts-nach-links).
	\item $X_2$: Ableitungsrichtung. Entweder $L$ (Linksableitung) oder $R$
		(Rechtsableitung).
	\item $k$: Anzahl Zeichen für Lookahead.
\end{itemize}

\subsection{Mehrdeutigkeiten}

Einige syntaktische Konstrukte lassen sich nur ungenau durch eine Grammatik beschreiben, eventuell entstehen Mehrdeutigkeiten in der Grammatik. Die Folge daraus ist, dass sich gewisse Sätze innerhalb einer Grammatik unterschiedlich interpretieren lassen. Mehrdeutigkeiten lassen sich manchmal innerhalb der Parserimplementation lösen, besser wäre allerdings eine eindeutige Grammatik. Mehrdeutigkeiten lassen sich nur mithilfe von Zusatzregeln vollständig eliminieren.

\textbf{Beispiel, mehrdeutiges If-Statement}

Grammatik:
\begin{align*}
	1.~STMT &\rightarrow if \; EXPR \; then \; STMT \\
	2.~STMT &\rightarrow if \; EXPR \; then \; STMT \; else \; STMT \\
\end{align*}
Beispielsatz: 
\[
	if \; E1 \; then \; if \; E2 \; then \; S1 \; else \; S2
\]

Dieser Satz kann unterschiedlich interpretiert werden:
\[
	(if \; E1 \; then \; ( \; if \; E2 \; then \; S1 \; else \; S2 \; ) \; )
\]
\[
	(if \; E1 \; then \; ( \; if \; E2 \; then \; S1 \; ) \; else \; S2 \; )
\]

Eine fehlende Regel für die Eindeutigkeit könnte lauten: Die 'else' Anweisung gehört immer zum am nächsten liegenden 'then'

\subsection{AST}

Der AST (Abstract Syntax Tree) ist eine vereinfachte Variante des Parse-Trees,
bei dem unnötige Details entfernt wurden. Mithilfe des ASTs werden
Code-Generierung sowie Optimierungen durchgeführt.


\subsection{Links- und Rechtsableitung}

Links- oder Rechts- beschreibt die Reihenfolge, in der die Nicht-Terminale durch
Terminale ersetzt werden. Bei der Linksableitung wird immer das erste
Nichtterminal von links gesehen als erstes ersetzt.


\subsection{Operator-Assoziativität}

Links- oder Rechtsassoziativität beschreibt die Reihenfolge, in welcher
Operatoren evaluiert werden.  Dies ist vor allem bei arithmetischen Ausdrücken
von Bedeutung. Als Beispiel: Addition ist linksassoziativ.
\[
	a + b + c + d = (((a + b) + c) + d)
\]

Potenzierung ist rechtsassoziativ.
\[
	a^{b^c} = a^{(b^{(c)})}
\]

Einige Operatoren können auch nichtassoziativ sein. Folgender Ausdruck ist
beispielsweise in vielen Programmiersprachen nicht erlaubt:
\[
	a < b < c
\]


\subsection{Eliminieren von Linksrekursion}

Einfache (direkte) Linksrekursionen können mithilfe des folgenden Algorithmus einfach entfernt
werden:

\begin{itemize}
	\item Es seien $A \rightarrow A\alpha_1 \mid A\alpha_2 \mid \ldots \mid A\alpha_m$ die
		linksrekursiven Regeln
	\item Es seien $A \rightarrow \beta_1 \mid \beta_2 \mid \ldots \mid \beta_n$ die nicht
		linksrekursiven Regeln
	\item Ersetze die erste Gruppe durch $A' \rightarrow \alpha_1A'\mid \alpha_2A' \mid \ldots \mid
		\alpha_mA'$
	\item Ersetze die zweite Gruppe durch $A \rightarrow \beta_1A' \mid \beta_2A' \mid \ldots \mid
		\beta_nA'$
	\item Füge folgende Regel hinzu: $A' \rightarrow \varepsilon$
\end{itemize}

Hinweis: Die Grammatik darf keine \textepsilon-Produktionen enthalten, ansonsten kann eine
versteckte Linksrekursion auftreten.

\textbf{Beispiel}
%
\begin{align*}
	A &\rightarrow Aa \mid bA \mid c
\end{align*}
%
\ldots{wird} zu
%
\begin{align*}
	A  &\rightarrow bAA' \mid cA' \\
	A' &\rightarrow aA' \mid \varepsilon 
\end{align*}


\subsection{Linksfaktorisierung}

Ein Parser liest den Tokenstream in der Regel von links nach rechts, deshalb
macht es Sinn, wenn der Parser sofort entscheiden kann, welche Regeln er
anwenden muss.  Wenn zwei Produktionen auf der rechten Seite jedoch einen
gemeinsamen Präfix haben, ist dies nicht möglich. Beispiel:
%
\begin{align*}
	STMT &\rightarrow if~COND~then~STMT \mid if~COND~then~STMT~else~STMT \mid OTHER
\end{align*}
%
Der gemeinsame Präfix ist in diesem Fall \textit{if COND then STMT}. Bei der
Linksfaktorisierung wird der Teil nach diesem Präfix in ein separate Regel
verschoben, damit der Entscheidungspunkt so weit verschoben wird, dass der
Parser eine direkte Entscheidung treffen kann.
%
\begin{align*}
	STMT &\rightarrow if~COND~then~STMT~OPTELSE \mid OTHER \\
	OPTELSE &\rightarrow else~STMT \mid \varepsilon
\end{align*}
%

\subsection{Parserbezeichnungen, Kurzbeschreibung}

\paragraph{LL(k) Parser}

\begin{itemize}
	\item \textbf{L}inks nach Rechts
	\item \textbf{L}inks Ableitungen
	\item \textbf{k} Tokens Lookahead
\end{itemize}
Ein LL-Parser ist ein Top-Down-Parser, der die Eingabe von Links nach rechts abarbeitet, um eine Linksableitung der Eingabe zu berechnen.

\paragraph{(S)LR(k)}

\begin{itemize}
	\item \textbf{S}imple
	\item \textbf{L}inks nach Rechts
	\item \textbf{R}echtsreduktion
	\item \textbf{k} Tokens Lookahead
\end{itemize}
Ein LR-Parser ist ein Bottom-Up-Parser für LR-Grammatiken. Bei diesen kontextfreien Grammatiken wird die Eingabe von links nach rechts abgearbeitet, um die Rechtsreduktion zum Startsymbol zu berechnen. Ein SLR Parser - im Umterschied zum LR Parser - versucht Shift/Reduce-Konflikte durch Betrachtung der FOLLOW-SETs besser zu lösen.

\subsection{Top-Down Parser} 

Ein Top-Down Parser beginnt beim Startsymbol der Grammatik und baut den
Parse-Baum von der Wurzel zu den Blättern sowie von links nach rechts auf.

\subsubsection{Recursive Descent Parser}

Eine speziell einfach implementierbare Variante eines Top-Down LL-Parsers ist
der Recursive Descent Parser. Für jedes Nichtterminal wird eine Methode
erstellt. Die Grammatikregeln werden als eine Folge von Aufrufen auf dieser
Methode realisiert. Die Terminale werden in diesen Methoden konsumiert.

Für einen Recursive Descent Parser benötigt man keine Parsetabelle, jedoch muss
mit Backtracking gearbeitet werden, was wiederum Performance-Einbussen bedeutet.

\textbf{Beispiel}

Gegeben ist folgende Grammatik:
%
\begin{align*}
	E &\rightarrow T \mid T + E \\
	T &\rightarrow int \mid int * T \mid (E)
\end{align*}
%
Daraus ergeben sich folgende Tokens:

\texttt{INT}, \texttt{OPEN}, \texttt{CLOSE}, \texttt{PLUS}, \texttt{TIMES}

Nachfolgend der C-Code eines einfachen Recursive Descent Parsers:

\begin{minted}[bgcolor=tango-bg,frame=lines,framesep=2mm,samepage=true]{c}
bool term(TOKEN tok) { return *next++ ==  tok; }

bool E1() { return T(); }
bool E2() { return T() && term(PLUS) && E(); }
bool E() {
    TOKEN *save = next;
    return (next = save, E1()) || (next = save, E2());
}
bool T1() { return term(INT); }
bool T2() { return term(INT) && term(TIMES) && T(); }
bool T3() { return term(OPEN) && E() && term(CLOSE); }
bool T() {
    TOKEN *save = next;
    return (next = save, T1()) || (next = save, T2()) || (next = save, T3());
}
\end{minted}

Das Problem bei diesem Algorithmus ist, dass er nicht alle Eingaben verarbeiten
kann. Mit der Eingabe \texttt{3 * 4} beispielsweise denkt der Parser nach der
Regel \texttt{T1} bereits, er hätte einen Match. Das folgende \texttt{*} kann er
jedoch nicht mehr zuordnen und failt. Dies liegt daran, dass die Grammatik nicht
linksfaktorisiert ist. Mit einer Linksfaktorisierung würde das Parsen aller
Eingaben klappen. Aber auch ohne Linksfaktorisierung können Recursive Descent
Parser mit erweitertem Backtracking implementiert werden, die mit allen
Grammatiken klarkommt.

Ein grosser Vorteil des Recursive Descent Parsers ist die Einfachheit der
Implementierung: Produktionen können quasi 1:1 im Code abgebildet werden.


\subsection{Bottom-Up Parser} 

Ein Bottom-Up Parser folgt einer umgekehrten Rechtsableitung.

\textbf{Beispiel}

Gegeben ist folgende Grammatik:
%
\begin{align*}
	E &\rightarrow T \mid T + E \\
	T &\rightarrow int \mid int * T
\end{align*}

Nachfolgend der Parse-Vorgang eines Bottom-Up Parsers für den Ausdruck $int * int + int$:
\[
	\begin{array}{|l|l|l|}
		\hline
		\textbf{Schritt} & \textbf{Ausdruck} & \textbf{Anzuwendende Regel} \\
		\hline
		1 & int * int + int & T \rightarrow int \\
		2 & int * \boldsymbol T + int & T \rightarrow int * T \\
		3 & \boldsymbol T + int & T \rightarrow int \\
		4 & T + \boldsymbol T & E \rightarrow T \\
		5 & T + \boldsymbol E & E \rightarrow T + E \\
		6 & \boldsymbol E & \\
		\hline
	\end{array}
\]

Wenn man diese Parse-Schritte von unten nach oben betrachtet, sieht man, dass immer das am weitesten
rechts stehende Non-Terminalsymbol (fett hervorgehoben) ersetzt wurde, es ist also eine
Rechtsableitung von unten nach oben ($\rightarrow$ Bottom-Up).


\subsection{FIRST und FOLLOW Sets}

\subsubsection{FIRST-Set}

Das FIRST-Set für ein Nicht-Terminal $X$ besteht aus allen Terminalsymbolen, die auf $X$ folgen können.
Dabei werden nachfolgende Terminalsymbole aufgelöst.

Man betrachtet dafür alle Produktionen, bei welchen $X$ auf der linken Seite steht:

% TODO Punkte 3 und 4 sind noch schlecht formuliert. Das muss noch besser gehen!
\begin{enumerate}
	\item Wenn auf der rechten Seite der Produktion als erstes ein Terminal steht, gehört dieses
		zum FIRST-Set von $X$.
	\item Wenn auf der rechten Seite der Produktion ein \textepsilon\ steht, gehört dieses zum
		FIRST-Set von $X$.
	\item Wenn auf der rechten Seite der Produktion als erstes ein Nicht-Terminal $Y$ steht, gehört
		das FIRST-Set von $Y$ (abzüglich allfälliger \textepsilon) zum FIRST-Set von $X$.
	\item Wenn dieses FIRST-Set von $Y$ ein \textepsilon\ enthält, berücksichtigt man auch alle Fälle,
		bei welchen $Y$ weggelassen wird. Wenn es keine solchen Fälle gibt, gehört auch
		\textepsilon\ zum FIRST-Set von $X$.
\end{enumerate}

\textbf{Beispiel}

Grammatik:
%
\begin{flalign*}
	S & \rightarrow Ax \mid By \mid z &\\
	A & \rightarrow 1CB \mid 2CB &\\
	B & \rightarrow 3B \mid C &\\
	C & \rightarrow 4 \mid \varepsilon &
\end{flalign*}
%
Lösungsweg:
%
\begin{enumerate}
	\item Das FIRST-Set von $C$ ist $\{4,\varepsilon\}$ gemäss Regeln 1 und 2.
	\item Das FIRST-Set von $B$ enthält gemäss Regel 1 $\{3\}$.
	\item Aufgrund der Produktion $B \rightarrow C$ und der Regel 3 enthält das FIRST-Set von $B$ auch
		das FIRST-Set von $C$, abzüglich des \textepsilon.
	\item Aufgrund der Regel 4 werden auch alle Produktionen berücksichtigt, bei denen das
		\textepsilon\ (und somit das $C$) "`überlesen"' wird. Da auf $C$ jedoch nichts mehr folgt,
		enthält FIRST(B) auch $\{\varepsilon\}$.
	\item Das FIRST-Set von $A$ ist gemäss Regel 1 $\{1,2\}$.
	\item Das FIRST-Set von $S$ enthält gemäss Regel 1 $\{z\}$.
	\item Das FIRST-Set von $S$ enthält gemäss Regel 3 das FIRST-Set von $A$ und $B$, abzüglich
		des \textepsilon\ in FIRST(B).
	\item Da FIRST(B) \textepsilon\ enthält, wird zusätzlich der Fall
		berücksichtigt, bei dem $B$ weggelassen wird. Da in der Produktion $S
		\rightarrow By$ auf das $B$ ein $y$ folgt, ist auch dieses in FIRST(S)
		enthalten. (Falls anstelle von $y$ ein Non-Terminal folgen würde, müsste man
		auch dieses auf"|lösen.)
\end{enumerate}
%
Obige Lösungsschritte resultieren in folgenden FIRST-Sets:
%
\begin{flalign*}
	& FIRST(S) = \{1,2,3,4,y,z\} &\\
	& FIRST(A) = \{1,2\} &\\
	& FIRST(B) = \{3,4,\varepsilon\} &\\
	& FIRST(C) = \{4,\varepsilon\} &
\end{flalign*}

\subsubsection{FOLLOW-Set}

Im Gegensatz zum FIRST-Set betrachtet man beim FOLLOW-Set alle Produktionen, bei
welchen $X$ auf der rechten Seite steht.

\begin{enumerate}
	\item Wenn $X$ das Startsymbol ist, gehört $\$$ zum FOLLOW-Set von $X$.
	\item Für jede Produktion $A \rightarrow \alpha X \beta$, füge
		FIRST(\textbeta) (ohne \textepsilon) zu FOLLOW(X) hinzu.
	\item Falls \textepsilon\ in FIRST(\textbeta) ist, füge FOLLOW(A) zu FOLLOW(X)
		hinzu.
	\item Für jede Produktion $A \rightarrow \alpha X$, füge FOLLOW(A) zu FOLLOW(X)
		hinzu.
\end{enumerate}

\textbf{Beispiel}

Grammatik:
%
\begin{flalign*}
	S & \rightarrow Ax \mid By \mid z &\\
	A & \rightarrow 1CB \mid 2CB &\\
	B & \rightarrow 3B \mid C &\\
	C & \rightarrow 4 \mid \varepsilon &
\end{flalign*}
%
Lösungsweg:
%
\begin{enumerate}
	\item Das FOLLOW-Set von $S$ enthält gemäss Regel 1 $\{\$\}$.
	\item Das FOLLOW-Set von $A$ enthält gemäss Regel 2 $\{x\}$.
	\item Das FOLLOW-Set von $B$ enthält gemäss Regel 2 $\{y\}$.
	\item Das FOLLOW-Set von $B$ enthält gemäss Regel 4 das FOLLOW-Set von $A$: $\{x\}$.
	\item Das FOLLOW-Set von $C$ enthält gemäss Regel 2 das FIRST-Set von $B$ ohne
		\textepsilon: $\{3, 4\}$.
	\item Das FOLLOW-Set von $C$ enthält gemäss Regel 3 das FOLLOW-Set von $A$:
		$\{x\}$.
	\item Das FOLLOW-Set von $C$ enthält gemäss Regel 4 das FOLLOW-Set von $B$:
		$\{x, y\}$.
\end{enumerate}
%
Obige Lösungsschritte resultieren in folgenden FOLLOW-Sets:
%
\begin{flalign*}
	& FIRST(S) = \{\$\} &\\
	& FIRST(A) = \{x\} &\\
	& FIRST(B) = \{x,y\} &\\
	& FIRST(C) = \{3,4,x,y\} &
\end{flalign*}


\subsection{LL(1) Parsetabellen}

Damit man eine deterministische Parsetabelle erstellen kann, braucht man eine
LL(1) Grammatik. Folgende Kriterien garantieren, dass eine Grammatik nicht LL(1)
ist:

\begin{itemize}
	\item Die Grammatik ist linksrekursiv
	\item Die Grammatik ist nicht linksfaktorisiert
	\item Die Grammatik ist mehrdeutig
\end{itemize}

Dies sind jedoch nicht die einzigen Kriterien. Um sicher zu gehen, dass eine
Grammatik LL(1) ist, muss man eine deterministische Parsetabelle aufstellen.

Die Parsetabelle ist vertikal mit den Nicht-Terminalsymbolen sowie mit $\$$
und horizontal mit den Terminalsymbolen beschriftet.

Folgender Algorithmus wird auf jede Produktion $A \rightarrow \alpha$ angewendet:

\begin{enumerate}
	\item Für jedes Terminalsymbol $t$ im FIRST-Set von \textalpha\ gilt: $T[A,t]
		= \alpha$.
	\item Falls \textepsilon\ im FIRST-Set von \textalpha\ ist, gilt für jedes
		Terminalsymbol im FOLLOW-Set von $A$: $T[A,t] = \alpha$.
	\item Falls \textepsilon\ im FIRST-Set von \textalpha\ ist und $\$$ im
		FOLLOW-Set von $A$ ist, gilt: $T[A,\$] = \alpha$.
\end{enumerate}

\textbf{Beispiel}

Grammatik:
%
\begin{flalign*}
	S & \rightarrow Ax \mid By \mid z &\\
	A & \rightarrow 1CB \mid 2CB &\\
	B & \rightarrow 3B \mid C &\\
	C & \rightarrow 4 \mid \varepsilon &
\end{flalign*}
%
Parsetabelle:
{%
	\newcommand{\WidestEntry}{$4 \mid \varepsilon$}%
	\newcommand{\stw}[1]{\makebox[\widthof{\WidestEntry}]{$#1$}}% stw -> set to widest
	\setlength{\mathindent}{0cm}
	\[
		\begin{array}{|c|c|c|c|c|c|c|c|c|}
			\hline
			& \stw{1} & \stw{2} & \stw{3} & \stw{4} & \stw{x} & \stw{y} & \stw{z} & \stw{\$} \\
			\hline
			S & Ax & Ax & By & By & & By & z & \\
			\hline
			A & 1CB & 2B & & & & & & \\
			\hline
			B & & & 3B & C & C & C & & \\
			\hline
			C & & & \varepsilon & 4 \mid \varepsilon & \varepsilon & \varepsilon & & \\
			\hline
		\end{array}
	\]
}%
Mithilfe dieser Parsetabelle sieht man, dass die gegebene Grammatik aufgrund
Mehrdeutigkeiten nicht LL(1) ist: Im Feld $T[C,4]$ gibt es zwei mögliche
Produktionen, welche man anwenden könnte.
